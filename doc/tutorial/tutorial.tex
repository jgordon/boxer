\documentclass[11pt]{article}

\usepackage{a4wide}
\usepackage{url}

\title{From shallow to deep natural language processing\\
       A hands-on tutorial}

\author{Johan Bos and Malvina Nissim}

\newtheorem{exercisebb}{\textbf{\underline{Exercise}:}}[section]

\newcommand{\bx}[1]{\begin{exercisebb} \rm #1\\}
\newcommand{\ex}{\end{exercisebb}}

\newcommand{\mnote}[1]
{
  \marginpar{\small #1}
  #1
}

\parindent 0pt
\parskip 6pt

\begin{document}
\maketitle

\begin{quote}
The aim of this tutorial is to provide hands-on experience in
open-domain text processing, covering the following topics:
tokenisation, part-of-speech tagging, named entity recognition,
parsing, semantic processing, and logical inference.  The tutorial
will comprise an overview of statistical modelling for natural
language processing tasks and brief introductions to the above topics,
followed by practical exercises.  It will be centered around the
state-of-the-art C\&C tools and Boxer.  No prerequisite knowledge is
required; nevertheless basic knowledge of shell commands in linux/unix
environments is a plus for getting the most out of this course.
\end{quote}

\tableofcontents

\clearpage
\section*{Getting Started}

The exercises of the tutorial require a running version of the C\&C
tools and Boxer.  These can be downloaded following the instructions at
the following URL:

\begin{quote}	
\url{http://svn.ask.it.usyd.edu.au/trac/candc/wiki}
\end{quote}

There are distributions for Linux, Mac OS or Windows CYGWIN operation
systems. The Linux or Mac OS environment is preferred. Ensure you got
the \underline{develo}p\underline{ment} \underline{version} of the
tools, which you get via the subversion repository.
Boxer requires the installation of SWI Prolog. 

\begin{quote}	
\url{http://www.swi-prolog.org}
\end{quote}

Note that all shell commands in this tutorial assume that your
directory is the \texttt{candc} directory (but it could have a
different name). Within this directory there should be the directories
\texttt{candc/bin} containing all binaries, and a directory
\texttt{candc/working}, for storing temporary results. It's handy to
create a new directory for the purpose of this tutorial:

\begin{quote}\tt
mkdir -p working/tutorial
\end{quote}

\vfill
\section*{Abbreviations}

\begin{description}
\item [CCG] Combinatory Categorial Grammar
\item [DRS] Discourse Representation Structure
\item [DRT] Discourse Representation Theory
\item [FOL] First-Order Logic
\item [NER] Named entity recognition
\item [POS] Part of speech
\end{description}



%
% Tokkie
%
\clearpage
\section{Tokenisation}\label{section:tokeniser}

\subsection*{Armchair material}

Tokenisation is typically the first thing you do in a pipeline of
natural language processing tools.  It involves usually two tasks,
which are often performed at the same time:

\begin{enumerate}
\item separating punctuation symbols from words; and
\item detecting sentence boundaries \marginpar{\small sentence boundary detection}
\end{enumerate}

Often white space is used as separation marks between tokens.
Sentences are usually separated by new lines. (But there are other
ways of doing this --- for instance by using an XML markup language.)

At first it might sound like a  tedious and rather trivial thing to do --- how
difficult can it really be to identify punctuation symbols such as
hyphens, commas, and full stops? The problem is that many \mnote{punctuation}
symbols are ambiguous in their use. To give an example: a hyphen can be used
in a football score, in a range of numbers, in a compound word, or to
divide a word at the end of line. 

For an accurate detection of sentence boundaries it is important to
distinguish full stops in the use of abbreviations, and when they mark
the end of a sentence (an abbreviation could be the end of the
sentence, in which case there is only one full stop!).


\subsection*{Hands-on stuff!}

First we need to install the tokeniser. This is done by invoking the shell command

\begin{quote}\tt
make bin/t
\end{quote}

in the \texttt{candc} directory. Now we're ready to play around with the tokeniser.


\bx{Tokenising} The following text is taken from Wikipedia
(\url{http://en.wikipedia.org/wiki/Kate_Bush}).  

\begin{quote}
Kate Bush (born 30 July 1958) is an English singer, songwriter, musician, and record producer. Her eclectic musical style and idiosyncratic lyrics have made her one of England's most successful solo female
performers of the past 30 years. Bush was signed by EMI at the age of 16 after being recommended by Pink Floyd's David Gilmour. In 1978, aged 19, she topped the UK charts for four weeks with her debut song
``Wuthering Heights'', becoming the first woman to have a UK number one with a self-written song.
\end{quote}

Copy it in a file
(called it for instance \texttt{katebush.txt} and store it in the
\texttt{working/tutorial} directory. Now tokenise it with the following command:

\begin{quote}\tt 
bin/t --input working/tutorial/katebush.txt
\end{quote}

This will send the output of the tokeniser to the screen. Compare the tokenised text with the
raw version.
\ex

\bx{Tokenisation options} 
Explore the various options (you can see
them with \texttt{bin/t --help} command).  Use the \texttt{--output}
option to redirect the output of the tokeniser to a file. And
\texttt{--mode rich} to generate character offsets.  
\ex

%
% POS
%
\clearpage
\section{Part-of-speech tagging}\label{section:tok}

\subsection*{Armchair material}

As we know, words can be used as \textit{nouns}, \textit{adjectives},
\textit{prepositions}, \textit{verbs}, and so on. These categories are
called parts of speech. Part-of-speech tagging, which you usually find
written simply as POS tagging, is the task of automatically labelling
each \textit{token} in the sentence with its \mnote{part of\\ speech}.

That's not so difficult, is it? Then let's consider the following sentence:

\begin{quote}
The function of sleep, according to one school of thought, is to consolidate memory.
\end{quote}

Here are the outputs of the same sentence labelled by two different taggers:

\begin{quote}
The-DET function-NOUN of-PREP sleep-NOUN ,-PUN according-VERB to-PREP one-DET school-NOUN of-IN thought-NOUN ,-PUN is-VERB to-PREP consolidate-VERB memory-NOUN .-PUN
\end{quote}

\begin{quote}
The-DET function-VERB of-PREP sleep-VERB ,-PUN according-VERB to-PREP one-DET school-NOUN of-IN thought-VERB ,-PUN is-VERB to-PREP consolidate-VERB memory-NOUN .-PUN
\end{quote}


As you can see, there are considerable differences between the output of these taggers. They are caused by two main issues:

\begin{itemize}
\item ambiguities in natural language
\item choice of tagset 
\end{itemize}

\paragraph{Ambiguity} Many words have more than just one syntactic category. Can you think of some?

\begin{quote}
The \textit{back} door = adjective\\
On my \textit{back} = noun\\
Win the voters \textit{back} = adverb\\
Promised to \textit{back} the bill = verb
\end{quote}

The POS tagging problem is to determine the POS tag for \textit{a
  particular instance} of a token.  When we perform the task of POS
tagging, we try to determine which of these syntactic categories
(i.e., POS tags) is the most likely for a particular use of a token in
a sentence.


\paragraph{Tagset} Which and how many tags should we use? This is an
issue researchers have been troubled with for a very very long time.
Already in ancient Greek times, Aristotle distinguished between
\textit{three} categories: nouns, predicates, and conjunctions.
Slightly later, a set of eight categories proposed by Dionysius Thrax,
in the 2nd century BC, was one that was maintained (more or less
unchanged) for a period of about two thousand years! And here is what Mark Twain 
says about the matter in his book ``The Awful German Language:''

\begin{quote}
\it There are ten parts of speech, and they are all troublesome.
\end{quote}

In the past decades, also due to the development of NLP techniques for
more sophisticated processing, the tagsets used have been extended to
comprise even up to 100 tags. Of course, tagsets are not always
independent of the language, and often also application dependent.

In this tutorial we use the Penn Treebank tagset (see Appendix), which comprises 36
different parts-of-speech plus punctuation. So, the sentence we have
seen above would be assigned the following more specific tags.

\begin{quote}
The-DT function-NN of-IN sleep-NN ,-, according-VBG to-TO one-CD school-NN of-IN thought-NN ,-, is-VBZ to-TO consolidate-VB memory-NN .-.
\end{quote}

Well, how good is it? The performance of current state-of-the-art POS
taggers shows an accuracy per token of around 97--98\%. This is
impressive! But what about accuracy \textit{per sentence}? How many
fully correctly tagged sentences do we get with a figure like this for
token accuracy?


\clearpage
\subsection*{Hands-on stuff!}

The POS-tagger is activated by the command \texttt{bin/pos}.
Here is an example:

\begin{quote}\tt
bin/pos --input working/tutorial/katebush.tok --model models/pos
\end{quote}

As you can see from the result, the output contains of the tokens, a
vertical bar, and the assigned part of speech.

\begin{quote}\tt
bin/pos --input working/tutorial/katebush.tok --model models/pos --output working/tutorial/katebush.pos
\end{quote}


\bx{Finding tagging mistakes}
Try to find mistakes in the tagged text, and try to give an explanation as to what caused the mistake.
\ex

\bx{Tagging homographs}\label{ex:homographs}
Homographs are words that spelled the same, but have different meanings (and usually different pronunciations).
Let's see what the POS-tagger makes of them! Try the following examples (thanks to A. Terry Bahill):
\begin{quote}
A cat with nine lives lives around the corner.\\
The dove dove into the deep grass.\\
I did not object to that object.\\
The landfill was so full it had to refuse refuse.
\end{quote}
\ex

\bx{Part of speech tags}
In the appendix of this document you will find a list of all tags used by the pos-tagger.
Complete the table by filling in the blanks.
\ex

%
% NER
%
\clearpage
\section{Named entity recognition}\label{section:ner}

\subsection*{Armchair material}

Named entities are phrases that contain \textbf{names} of any type
which could be of interest for natural language processing tasks.
These could be names of, as in most standard approaches,
\textit{persons}, \textit{organisations}, \textit{locations},
\textit{times} and \textit{quantities}. But they could also be
\textit{genes}, \textit{proteins}, \textit{cell types} and any other
sort of interesting entity for those who want to mine biological data.
Or for someone building an application concerning published works, the
names entities of interest could comprise \textit{films},
\textit{books}, \textit{paintings} and the like.

So, first one has to find out \textit{which entities} are of interest
for our field and task, and then \textit{assign a label} to them. In
other words, there are two steps to named entity recognition (NER):

\begin{enumerate}
\item \textit{detecting} named entities
\item \textit{classifiying} named entities
\end{enumerate}

However, these two step are often performed in tandem and difficult to
distinguish as two separate phases.

So, how are we going to find these entities? Well, one way would be to
have very long precompiled lists of names (usually called
\textit{gazetteers}), and in some cases they do not work that bad. For
example, we could compile lists of countries, cities, rivers, and so
on, and the list might be quite exhaustive. But whereas this holds for
more ``closed" classes, this is not true for more open ones, such as
organisations. New companies are born (and die) every day, and it
would be not only ridiculously time-consuming but perhaps also quite
pointless and foolish to try list them all.

But the problem is not only completeness.  \textit{Ambiguity} (yet
again) is the other reason why gazetteers cannot fully work. Is
``Washington" a person or a place? Does ``Ericson'' refer to a company
or a person?  Is ``1984'' a time expression or a measure phrase? Well,
it depends on the context, of course. What the NER module is supposed
to do is exactly dealing with this problem: tagging entities in
context (a sentence).

The NE tagger that we use in our tutorial takes care of seven
different classes, as shown in the table below.

\begin{center}
\begin{tabular}{|r|l|l|}
\hline
\textbf{NE} & \textbf{Description} & \textbf{Example}\\
\hline
\hline
PER & named person & Kate Bush\\
DAT & date expressions & January 16\\
TIM & time expressions & 5 p.m.\\
LOC & named locations & England\\
MON & monetary expressions & 50 million dollar\\
ORG & organisation & Pink Floyd \\
PCT & percentage & 10\% \\
\hline
\end{tabular}
\end{center}



%Prefixed by B (begin) or I (in), E (end). 
%O = Out.

\clearpage
\subsection*{Hands-on stuff!}

\bx{Using the NE tagger}
Try this to start the tagger:

\begin{quote}\tt
bin/ner --input working/tutorial/katebush.pos --model models/ner
\end{quote}
\ex

\bx{Finding tagging mistakes}
Try to find mistakes in the tagged text, and try to give an explanation as to what caused the mistake.
\ex

\bx{Training a tagger}
Add a new NE class to the training data. Use \texttt{bin/train\_ner} to create a new model. Then evaluate the new model on a different portion of data.
\ex

%
% SYNTAX
%
\clearpage
\section{Syntax}\label{section:parser}

\subsection*{Armchair material}

A parser assigns syntactic structure to a string, based on a grammar
and lexicon.  We follow \mnote{Combinatory Categorial Grammar} in
defining the lexicon and grammar for English. CCG is a lexicalised
theory of grammar: it has many lexical categories, but few grammar
rules (combinatory rules as they are called in CCG). 

In a CCG grammar, categories are assigned to words and expressions.
Categories can be simple (S for sentence, NP for noun phrase, N for
noun, PP for prepositional phrase) or combined. The combined
categories are composed by the use of the slashes ``$\backslash$'' and
``/''.  For instance, S$\backslash$NP for an intransitive verb, and
N/N is the category for an adjective. The slashes indicate
directionality: the $\backslash$ tells to look at the immediate left,
the / at the immediate right. So the N/N category encodes the
information that it is looking for an N on its right. The
S$\backslash$NP category encodes the information that it is looking
for an NP on its left.

CCG uses combinatory rules to combine smaller phrases (categories)
into larger ones. There is only a dozen number of such rules. The most
common ones are application and composition. The rules can be defined
using the following templates:

\begin{quote}
\begin{verbatim}
X/Y   Y          Y   X\Y         X/Y   Y/Z         Y\Z  X\Y
-------[fa]      -------[ba]     ---------[fc]     --------[bc]
   X                X               X/Z              X\Z
\end{verbatim}
\end{quote}

Here the abbreviations fa and ba stand for forward and backward
applicaton respectively, and fc and bc for forward and backward
composition. There are also other rules that deal with type shifting,
coordination, and punctuation, depending on the version of CCG theory
you're using.


\subsection*{Hands-on stuff!}

To produce syntactic structure (in the form of CCG derivations) we use
the C\&C parser, a statistical parser for CCG, trained on a large
database of CCG derivations (CCGbank). For convenience we will use a
combined program that can be started via \texttt{bin/candc} that also
performs part-of-speech tagging and named entity recognition. 

The parser expects one sentence per line, and the sentences need to be tokenised. 
Here is how you run the parser:

\begin{quote}\tt
bin/candc --input <FILE> --models models/boxer --candc-printer boxer
\end{quote}

What do these options mean?  The \texttt{--models
models/boxer} option selects a pre-trained statistical model of the
grammar (it's a large model, that's why it takes quite a bit of time
to load it --- if you run the parser in server mode you don't face
this problem).  The \texttt{--candc-printer boxer} option ensures that
the output is displayed in CCG derivations. It is possible to redirect
the output to a file via the \texttt{--output <FILE>} option.

\bx{Parsing}
Try the parser using the command above on a (tokenised) text. You can
also try to the parser on a raw text, and find out why tokenisation is 
important!
\ex

\bx{Inspecting the output}

Create a file \texttt{working/tutorial/syn1.txt} with the following sentence:

\begin{quote}\tt
Bill Gates stepped down as chief executive officer of Microsoft\\
on January 12, 2000.
\end{quote}

Tokenise it, then parse it. What are the lexical categories? 
And the combinatorial rules?
\ex

\bx{Parsing homographs}
Try the parser on the sentences with homographs given in Exercise~\ref{ex:homographs} on Page~\pageref{ex:homographs}.
\ex




%
% SEMANTICS
%
\clearpage
\section{Semantics}\label{section:semantics}

In this part of the tutorial we will see how we can construct semantic
representations on the basis of the syntactic derivations produced by
the parser. The tool we're going to use for this purpose is
\textbf{Boxer}.  (If you haven't compiled \mnote{Boxer} yet, type:
\texttt{make bin/boxer} on the command line.)

\subsection*{Armchair material}

The semantic representations that we will produce are known as
Discourse Representation Structures (DRSs), which are proposed in
\mnote{Discourse Representation Theory}, a formal theory of natural
language meaning dealing with a wide variety of linguistic phenomena.

Simply put, a DRS consists of a pair of discourse referents
(also known as the domain of the DRS) and a set of conditions.  A
discourse referent denotes an entity, a condition constrains the
interpretation of this entity. DRSs are recursive structures --- a DRS
might contain other DRSs constructed with the logical operators
negation, disjunction, or implication. 

DRSs can be converted to formulas of first-order logic.  Discourse
referents are interpreted as existential or universal quantifiers,
depending on the position of the DRS in which it occurs is embedded in
other DRSs or not. The structure of DRSs is used in pronoun
resolution: discourse referents in embedded DRSs are not accessible
for pronouns. 



\subsection*{Hands-on stuff!}

\bx{Boxing}
Create a file \texttt{working/tutorial/sem1.txt} with the following three sentences:

\begin{quote}\tt
Bill saw a busy manager.\\
Bill saw every busy manager.\\
Bill saw no busy manager.
\end{quote}

Use the tokeniser (Section~\ref{section:tokeniser}) to
tokenise these sentence in a file \texttt{working/tutorial/sem1.tok}.
Then produce a CCG derivation with the parser
(Section~\ref{section:parser}), and redirect the output in the file
\texttt{working/tutorial/sem1.ccg}. Then run Boxer with the following
command:

\begin{quote}\tt
bin/boxer --input working/tutorial/sem1.ccg --box --resolve
\end{quote}

This should produce DRSs both in Prolog format and
pretty-printed boxes. 
\ex

\bx{DRS structure}
Compare the DRS structures of the three
DRSs. The \texttt{+} indicates a conjunction of DRSs, the \texttt{>}
an implication, and \texttt{--} a negation.  Try to paraphrase the
content of the DRSs in English.  
\ex

\bx{Output formats}
Run the same command as before but
now add the option \texttt{--format no} and then again with
\texttt{--format xml}.
\ex

\bx{Boxing}
Create a file \texttt{working/tutorial/sem2.txt} with the following three lines:

\begin{quote}\tt
<META>text1\\
Mr.~Jones bought an expensive car.\\
He is a busy manager.
\end{quote}

Tokenise this file, then parse it and store the output in a file
\texttt{working/tutorial/sem2.ccg}.  Now run Boxer on this file, once
without the \texttt{--resolve} option, and once with, then compare the
results:

\begin{quote}\tt
bin/boxer --input working/tutorial/sem2.ccg --format no --box\\
bin/boxer --input working/tutorial/sem2.ccg --format no --box --resolve
\end{quote}
\ex

The \texttt{<META>} tag causes the sentences following it to be
analysed in one DRS, rather than in a separate DRS for each DRS. You
can have more than one \texttt{<META>} tags in an input file.



\bx{Thematic Roles}
Boxer employs a neo-Davidsonian approach to event semantics. This
means that verbs introduce discourse referents denoting events, which
are linked to participants of the event by thematic roles. By default,
Boxer uses a simple set of roles: agent, patient, and theme.  Run Boxer
on the example with the option \texttt{--roles verbnet}. Can you spot
the differences?
\ex

\bx{Discourse Relations} If you run Boxer with the option
\texttt{--theory sdrt} you will see that the output is produced in
the form of Segmented Discourse Representation Structures, following
SDRT.  Boxer has currently limited ways to automatically assign
discourse structure and rhetorical relations to texts (it assigns
almost always the same discourse relation between segments), but this
will hopefully change in the near future.
\ex

\bx{First-order Logic} In the next section we will see how we can
perform various inference tasks on DRSs.  The way this works is by
translating DRSs into formalus of first-order logic, that are then
given to automated theorem provers and model builders.  If you run
Boxer with the option \texttt{--semantics fol} you will see the
translation into first-order logic.  
\ex

%
% INFERENCES
%
\clearpage
\section{Inference}\label{section:inference}

Here we will see how we can use techniques from automated deduction to
draw logical inferences from texts. We will do this with the help of
\textbf{Nutcracker}, a system for recognising textual entailment.

\subsection*{Armchair material}

The two tools we are going to use are a theorem prover and a model
builder for first-order logic (FOL).  We will
translate the DRSs generated by Boxer to FOL and then give it to the
theorem prover and model builder.  As the name suggests, a
\mnote{theorem prover} attempts to find out whether a given input
formula is a theorem. In other words, it checks whether the input is a
validity, that is, true in all possible models. On the other hand, a \mnote{model
builder} attempts to construct a model for a given
input.

How can we make use of these tools? Imagine a text is inconsistent,
such as the following sentence:

\begin{quote}
Mr Jones is a woman.
\end{quote}

Say we produced a semantic representation for this sentence, and
generated background knowledge that Mr entails a male human entity,
and woman a female human entity, and that male and female are disjoint
properties. We all put this in one big formula -- let's call it
$\phi$.  Now when given $\phi$ to a model builder, the model builder
will fail to find a model, because there isn't one satisfying the
input ($\phi$ doesn't have a model, or $\phi$ is not satisfiable).  It
might be that in such cases the model builder is going on forever (or
for a very long time) trying to find a model.

Therefore we also use a theorem prover. But we don't give it
$\phi$. We give it the negation, $\lnot \phi$.  Why? Well think about
it: if $\phi$ isn't true in any model, then the negation of $\phi$
must be true in all models. And that's precisely what theorem provers
are good in finding out.

In sum: the theorem prover and model builder are complementary
tools. We always use them together. When we give $\phi$ to the model
builder, we give $\lnot \phi$ to the theorem prover. If the theorem
prover finds a result, we know that the input text is inconsistent,
and if the model builder finds a result, we know that the text is
consistent. So basically, we're now able to perform \mnote{consistency
checking} for texts.

\clearpage
\subsection*{Hands-on stuff!}

\bx{Consistency checking}
Make a directory \texttt{working/tutorial/rte}. Create two text files in this
directory: a file named \texttt{t}, and a file named \texttt{h}. Make the contents 
of the first file the sentence ``Bill Gates bought a car.'', and the that of the second
``Bill Gates bought no car.''. Then run Nutcracker:

\begin{quote}\tt
bin/nc --dir working/tutorial/rte --info
\end{quote}
\ex

\bx{Background knowledge}
Nutcracker calculates background knowledge using the \mnote{WordNet} lexical database. It uses this
background knowledge while it draws inferences. Let's look at some examples:

\begin{quote}\tt
t:John has a dog.\\
h:John has an animal.
\end{quote}

\begin{quote}\tt
t:John has an animal.\\
h:John has a dog.
\end{quote}

\begin{quote}\tt
t:John likes no animal.\\
h:John likes a dog.
\end{quote}

What does Nutcracker predict for these three textual entailment pairs? Find out what the background knowledge ontology
looks like by viewing the contents of the file \texttt{mwn.pl} in the directory of each example.
\ex


\bx{Looking under the hood} After using Nutcracker, have a look at
files generated for drawing inferences.  They are stored in the
directory of each example, and bear names with the extensions
\texttt{*.in} and \texttt{*.out}. These files contain the last
generated input for and output from a theorem prover or model builder.
\ex

\clearpage
\section*{Appendix: Part of Speech (POS) tagset}
\begin{tabular}{|r|l|l|}
\hline
\textbf{Tag} & \textbf{Desciption (Penn Treebank tagset)} & \textbf{Example}\\
\hline
\hline
    CC &    Coordinating conjunction  &  \hspace*{40mm} \\ \hline
    CD &    Cardinal number  & \\ \hline
    DT &    Determiner  & \\ \hline
    EX &    Existential ``there'' & \\ \hline
    FW &    Foreign word  & \\ \hline
    IN &    Preposition or subordinating conjunction  & \\ \hline
    JJ &    Adjective  & \\ \hline
    JJR &    Adjective, comparative  & \\ \hline
    JJS &    Adjective, superlative  & \\ \hline
    LS &    List item marker  & \\ \hline
    MD &    Modal  & \\ \hline
    NN &    Noun, singular or mass  & \\ \hline
    NNS &    Noun, plural  & \\ \hline
    NNP &    Proper noun, singular  & \\ \hline
    NNPS &    Proper noun, plural  & \\ \hline
    PDT &    Predeterminer  & \\ \hline
    POS &    Possessive ending  & \\ \hline
    PRP &    Personal pronoun  & \\ \hline
    PRP\$ &    Possessive pronoun  & \\ \hline
    RB &    Adverb  & \\ \hline
    RBR &    Adverb, comparative  & \\ \hline
    RBS &    Adverb, superlative  & \\ \hline
    RP &    Particle  & \\ \hline
    SYM &    Symbol  & \\ \hline
    TO &    ``to''  & \\ \hline
    UH &    Interjection  & \\ \hline
    VB &    Verb, base form  & \\ \hline
    VBD &    Verb, past tense  & \\ \hline
    VBG &    Verb, gerund or present participle  & \\ \hline
    VBN &    Verb, past participle  & \\ \hline
    VBP &    Verb, non-3rd person singular present  & \\ \hline
    VBZ &    Verb, 3rd person singular present  & \\ \hline
    WDT &    Wh-determiner  & \\ \hline
    WP &    Wh-pronoun  & \\ \hline
    WP\$ &     Possessive wh-pronoun  & \\ \hline
    WRB &     Wh-adverb  & \\
\hline
\end{tabular}

\end{document}
